%%%%%%%% ICML 2020 EXAMPLE LATEX SUBMISSION FILE %%%%%%%%%%%%%%%%%

\documentclass{article}

% Recommended, but optional, packages for figures and better typesetting:
\usepackage{microtype}
\usepackage{graphicx}
\usepackage{subfigure}
\usepackage{booktabs} % for professional tables
\usepackage{mathtools}
\usepackage{physics}

% hyperref makes hyperlinks in the resulting PDF.
% If your build breaks (sometimes temporarily if a hyperlink spans a page)
% please comment out the following usepackage line and replace
% \usepackage{icml2020} with \usepackage[nohyperref]{icml2020} above.
\usepackage{hyperref}

% Attempt to make hyperref and algorithmic work together better:
\newcommand{\theHalgorithm}{\arabic{algorithm}}

% Use the following line for the initial blind version submitted for review:
% \usepackage{icml2020}

% If accepted, instead use the following line for the camera-ready submission:
\usepackage[accepted]{icml2020}

% The \icmltitle you define below is probably too long as a header.
% Therefore, a short form for the running title is supplied here:
\icmltitlerunning{Optimal flocking with a genetic algorithm}

\begin{document}

\twocolumn[
\icmltitle{Optimal boid flocking with a genetic algorithm}

% It is OKAY to include author information, even for blind
% submissions: the style file will automatically remove it for you
% unless you've provided the [accepted] option to the icml2020
% package.

% List of affiliations: The first argument should be a (short)
% identifier you will use later to specify author affiliations
% Academic affiliations should list Department, University, City, Region, Country
% Industry affiliations should list Company, City, Region, Country

% You can specify symbols, otherwise they are numbered in order.
% Ideally, you should not use this facility. Affiliations will be numbered
% in order of appearance and this is the preferred way.
% \icmlsetsymbol{equal}{*}

\begin{icmlauthorlist}
\icmlauthor{Frederik J. Mellbye}{to}
\end{icmlauthorlist}

\icmlaffiliation{to}{Institute for Computational and Mathematical Engineering, Stanford University, Stanford, USA}

\icmlcorrespondingauthor{Frederik J. Mellbye}{frederme@stanford.edu}

% You may provide any keywords that you
% find helpful for describing your paper; these are used to populate
% the "keywords" metadata in the PDF but will not be shown in the document
\icmlkeywords{Machine Learning, ICML}

\vskip 0.3in
]

% this must go after the closing bracket ] following \twocolumn[ ...

% This command actually creates the footnote in the first column
% listing the affiliations and the copyright notice.
% The command takes one argument, which is text to display at the start of the footnote.
% The \icmlEqualContribution command is standard text for equal contribution.
% Remove it (just {}) if you do not need this facility.

\printAffiliationsAndNotice{}  % leave blank if no need to mention equal contribution
% \printAffiliationsAndNotice{\icmlEqualContribution} % otherwise use the standard text.

\begin{abstract}
A genetic evolution-based approach is used to determine the coefficients of a boids model for optimal flocking. The model is inspired by the original Reynolds model, and adds predators. The results show that ...
\end{abstract}

\section{Introduction}
\label{sec:introduction}
What are boids?

Applications of boids framework:
- Helped understand animal flocking (e.g. how do birds determine where to fly when)
- CGI (movies, games)
- Collective motion of drone/vehicle swarms. Very feasible, because each agent measures and reacts to surroundings on its own, without the need of a central controller that controls all boids. In simulations, this allows parallelization.

\section{Background}
\label{sec:background}
Reynolds \cite{reynolds:1987:CG}.

Optimization of boids \cite{ntnu}.

\section{Approach}
\label{sec:approach}
\subsection{Boids system}
\begin{figure*}
  \centering
  \includegraphics[width=.25\linewidth]{images/separation.png}
  \includegraphics[width=.25\linewidth]{images/alignment.png}
  \includegraphics[width=.25\linewidth]{images/cohesion.png}
  \caption{Steering forces on prey boids. From the left: Separation, alignment and cohesion \cite{reynolds:1987:CG}.}
  \label{fig:preyforces}
\end{figure*}

Prey boids have the three original steering forces; cohesion, separation and alignment \cite{reynolds:1987:CG}. These are depicted in Figure~\ref{fig:preyforces}.

The cohesion force $F_c$ on boid $b_j$ is given by
\begin{align}
  F_c = c_c \frac{1}{n-1} \sum_{i \neq j}^{n} x_i - x_j
\end{align}
where $c_c$ is a corresponding coefficient on the force, and $x_i$ is the position of $b_i$. It draws boids closer to the general position of neighboring boids.

Similarly, the separation force is given by
\begin{align}
  F_s = c_s \frac{1}{n-1} \sum_{i \neq j}^{n} \frac{x_i - x_j}{\norm{x_i - x_j}_2}
\end{align}
This force only considers boids that are very close, and ensures boids do not fly too close to each other. In some sense this simulates that the boids have some size and avoid crashing.

Finally, the alignment force is
\begin{align}
  F_a = c_a \frac{1}{n-1} \sum_{i \neq j}^{n} v_i - v_j
\end{align}
where $v_i$ is the velocity vector of $b_i$. Boids steer towards the average heading of neighboring boids.

The forces are combined to produce a total force, which is clipped, simulating how animals only can turn at certain rates. The total force is therefore given by
\begin{align*}
  F = \max(F_c + F_s + F_a, F_\text{max})
\end{align*}
The boid masses are normalized, so the force is the acceleration. We also have a time step $\Delta t = 1$. Therefore, the velocity and position updates are
\begin{align*}
  v^{(k+1)} &= v^{(k)} + F \\
  x^{(k+1)} &= x^{(k)} + \frac{v^{(k+1)}}{\norm{v^{(k+1)}}_2}
\end{align*}
Note that the velocity is normalized. This paper assumes that all boids fly at their max speed $v_\text{max}$ at all times.

\begin{figure*}
  \centering
  \includegraphics[width=.25\linewidth]{images/chase.png}
  \includegraphics[width=.25\linewidth]{images/avoid.png}
  \caption{Hunt and flee steering forces on predator boids. Modified versions of images from \cite{reynolds:1987:CG}.}
  \label{fig:predatorforces}
\end{figure*}

\subsection{Genetic algorithm}
Genetic algorithm approaches commonly have three components; selection, crossover and mutation \cite{kochenderfer}. The fashion in which these are performed commonly differs based on the particular application and the way the chromosomes are encoded.



\begin{table*}
  \centering
  \caption{Parameters used for the genetic algorithm and boids simulation}
  \label{tbl:params}
\begin{tabular}{*3l} \toprule
\emph{Name}       & \emph{Symbol} & \emph{Value} \\ \midrule
           &        &       \\
Separation range           &        &       \\
Prey force ranges         &        &       \\
Kill range          &        &       \\
Hunting coefficient & $c_h$ &       \\
Flee coefficient &  $c_f$ &       \\
Predator $v_\text{max}$ &       & 1.1      \\
Prey $v_\text{max}$ &        & 1.0      \\
           &        & \\\bottomrule
\end{tabular}
\end{table*}

\subsection{Computational considerations}
For performance reasons, the implementation of the boids framework and genetric algorithm was done in C++.

With a naive implementation the computation of the steering forces is $\mathcal{O}(n^2)$. For each boid, the positions of all other boids are required. This is clearly not feasible for simulations with many boids and many generations of evolution.

To reduce the computational load, each boid locally keeps track of it's nearest neighbors and only use these to compute steering. This is updated with some frequency.

\section{Results}
\label{sec:results}

\section{Conclusion}
\label{sec:conclusion}

\section{Future directions}
\label{sec:future}
More realistic physics simulation -> interesting applications. E.g. drone swarms. Applications in e.g. defense systems.

Also optimize predator behaviour (this could in turn further optimize prey behaviour. Turns into a min-max problem.)

Add obstacles and potentially add some goal area to get to.


% In the unusual situation where you want a paper to appear in the
% references without citing it in the main text, use \nocite

\bibliography{references}
\bibliographystyle{icml2020}

%%%%%%%%%%%%%%%%%%%%%%%%%%%%%%%%%%%%%%%%%%%%%%%%%%%%%%%%%%%%%%%%%%%%%%%%%%%%%%%
%%%%%%%%%%%%%%%%%%%%%%%%%%%%%%%%%%%%%%%%%%%%%%%%%%%%%%%%%%%%%%%%%%%%%%%%%%%%%%%


\end{document}


% This document was modified from the file originally made available by
% Pat Langley and Andrea Danyluk for ICML-2K. This version was created
% by Iain Murray in 2018, and modified by Alexandre Bouchard in
% 2019 and 2020. Previous contributors include Dan Roy, Lise Getoor and Tobias
% Scheffer, which was slightly modified from the 2010 version by
% Thorsten Joachims & Johannes Fuernkranz, slightly modified from the
% 2009 version by Kiri Wagstaff and Sam Roweis's 2008 version, which is
% slightly modified from Prasad Tadepalli's 2007 version which is a
% lightly changed version of the previous year's version by Andrew
% Moore, which was in turn edited from those of Kristian Kersting and
% Codrina Lauth. Alex Smola contributed to the algorithmic style files.
