%%%%%%%%%%%%%%%%%%%%%%%%%%%%%%%%%%%%%%%%%
% Beamer Presentation
% LaTeX Template
% Version 1.0 (10/11/12)
%
% This template has been downloaded from:
% http://www.LaTeXTemplates.com
%
% License:
% CC BY-NC-SA 3.0 (http://creativecommons.org/licenses/by-nc-sa/3.0/)
%
%%%%%%%%%%%%%%%%%%%%%%%%%%%%%%%%%%%%%%%%%

%----------------------------------------------------------------------------------------
%	PACKAGES AND THEMES
%----------------------------------------------------------------------------------------

\documentclass{beamer}

\mode<presentation> {

% The Beamer class comes with a number of default slide themes
% which change the colors and layouts of slides. Below this is a list
% of all the themes, uncomment each in turn to see what they look like.

%\usetheme{default}
%\usetheme{AnnArbor}
%\usetheme{Antibes}
%\usetheme{Bergen}
%\usetheme{Berkeley}
%\usetheme{Berlin}
%\usetheme{Boadilla}
%\usetheme{CambridgeUS}
%\usetheme{Copenhagen}
%\usetheme{Darmstadt}
%\usetheme{Dresden}
%\usetheme{Frankfurt}
%\usetheme{Goettingen}
%\usetheme{Hannover}
%\usetheme{Ilmenau}
%\usetheme{JuanLesPins}
%\usetheme{Luebeck}
\usetheme{Madrid}
%\usetheme{Malmoe}
%\usetheme{Marburg}
%\usetheme{Montpellier}
%\usetheme{PaloAlto}
%\usetheme{Pittsburgh}
%\usetheme{Rochester}
%\usetheme{Singapore}
%\usetheme{Szeged}
%\usetheme{Warsaw}

% As well as themes, the Beamer class has a number of color themes
% for any slide theme. Uncomment each of these in turn to see how it
% changes the colors of your current slide theme.

%\usecolortheme{albatross}
%\usecolortheme{beaver}
%\usecolortheme{beetle}
%\usecolortheme{crane}
%\usecolortheme{dolphin}
%\usecolortheme{dove}
%\usecolortheme{fly}
%\usecolortheme{lily}
%\usecolortheme{orchid}
%\usecolortheme{rose}
%\usecolortheme{seagull}
%\usecolortheme{seahorse}
%\usecolortheme{whale}
%\usecolortheme{wolverine}

%\setbeamertemplate{footline} % To remove the footer line in all slides uncomment this line
%\setbeamertemplate{footline}[page number] % To replace the footer line in all slides with a simple slide count uncomment this line

%\setbeamertemplate{navigation symbols}{} % To remove the navigation symbols from the bottom of all slides uncomment this line
}

\usepackage{graphicx} % Allows including images
\usepackage{booktabs} % Allows the use of \toprule, \midrule and \bottomrule in tables

%----------------------------------------------------------------------------------------
%	TITLE PAGE
%----------------------------------------------------------------------------------------

\title[Optimal flocking with genetic algorithm]{Optimal flocking of boids using a genetic algorithm} % The short title appears at the bottom of every slide, the full title is only on the title page

\author{Frederik J. Mellbye} % Your name
\institute[SU] % Your institution as it will appear on the bottom of every slide, may be shorthand to save space
{
Stanford University \\ % Your institution for the title page
\medskip
\textit{frederme@stanford.edu} % Your email address
}
\date{\today} % Date, can be changed to a custom date

\begin{document}

\begin{frame}
\titlepage % Print the title page as the first slide
\end{frame}

% \begin{frame}
% \frametitle{Overview} % Table of contents slide, comment this block out to remove it
% \tableofcontents % Throughout your presentation, if you choose to use \section{} and \subsection{} commands, these will automatically be printed on this slide as an overview of your presentation
% \end{frame}

%----------------------------------------------------------------------------------------
%	PRESENTATION SLIDES
%----------------------------------------------------------------------------------------

%------------------------------------------------

%------------------------------------------------

\begin{frame}
\frametitle{Boids \cite{Reynolds}}
\begin{itemize}
\item Artificial Life program
\item Used to simulate flocking
\item Simple behavior rules
\end{itemize}
\begin{figure}
  \includegraphics[width=.9\linewidth]{images/flock.jpg}
  \caption{From \url{http://www.columbia-audubon.org/birds-in-big-numbers-flocks-of-blackbirds-and-starlings/}}
\end{figure}
\end{frame}

%------------------------------------------------

\begin{frame}
\frametitle{Boid rules}
\begin{columns}[c] % The "c" option specifies centered vertical alignment while the "t" option is used for top vertical alignment

\column{.55\textwidth} % Left column and width
\begin{block}{Separation}
Boids steer away from each other when too close, to avoid crashing into each other.
\end{block}
\begin{block}{Alignment}
Steer towards the average heading of local flockmates, aligning their directions.
\end{block}

\begin{block}{Cohesion}
Steer towards the average position of local flockmates.
\end{block}

\column{.4\textwidth} % Right column and width
\begin{figure}
  \centering
  \includegraphics[width=.6\linewidth]{images/separation.png}
  \includegraphics[width=.6\linewidth]{images/alignment.png}
  \includegraphics[width=.6\linewidth]{images/cohesion.png}
\end{figure}

\end{columns}
\end{frame}

%------------------------------------------------

\begin{frame}
\frametitle{Additions: Predators and boid removal}

\begin{itemize}
\item Predators: chase nearest prey boid
\begin{figure}
  \includegraphics[width=.25\linewidth]{images/chase.png}
\end{figure}
\item Prey boids are removed if a predator gets sufficiently close
\item Predators flee if too many prey boids in close proximity
\begin{figure}
  \includegraphics[width=.25\linewidth]{images/avoid.png}
\end{figure}
\end{itemize}

\end{frame}

%------------------------------------------------

%------------------------------------------------
\begin{frame}
\frametitle{Computing the force}

\begin{itemize}
  \item
  The force on a boid is a weighted sum of the steering forces:
  \begin{align*}
    \text{force} = c_1 \times \text{Separation} + c_2 \times \text{Alignment} + c_3 \times \text{Cohesion}
  \end{align*}
  \item
  Chromosome of $i$-th boid:
  \begin{align*} \mathbf{c^{(i)}} =
    \begin{bmatrix}
      c_1 & c_2 & c_3
    \end{bmatrix}
  \end{align*}
\end{itemize}

\end{frame}
%------------------------------------------------

\begin{frame}
\frametitle{This project}

\begin{itemize}
  \item Goal: Find design (set of boid coefficients $\mathbf{x} = \{ c^{(i)} \}_{i=1}^{N}$) that minimizes the number of deceased prey boids.
  \item This will be done using a genetic algorithm. For each generation, real-valued crossover is performed on the survivors to form the next generation. Mutation is added in the form of $\mathcal{N}(0,\sigma^2)$ noise.
  \item Hypothesis: Over many generations, the algorithm will find good parameters for flocking.
\end{itemize}

\end{frame}

%------------------------------------------------

\begin{frame}
\frametitle{Sample runs for different designs}
See attached video in the repository.
\end{frame}

%------------------------------------------------

\begin{frame}
\frametitle{References}
\footnotesize{
\begin{thebibliography}{99} % Beamer does not support BibTeX so references must be inserted manually as below
\bibitem[Reynolds, 1987]{Reynolds} Reynolds, C. W. (1987)
\newblock Flocks, Herds, and Schools: A Distributed Behavioral Model, in Computer Graphics
\newblock 21(4) (\emph{SIGGRAPH '87 Conference Proceedings}) pages 25-34.
\end{thebibliography}
}
\end{frame}

\end{document}
